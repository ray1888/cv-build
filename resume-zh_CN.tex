% !TEX TS-program = xelatex
% !TEX encoding = UTF-8 Unicode
% !Mode:: "TeX:UTF-8"

\documentclass{resume}
\usepackage{zh_CN-Adobefonts_external} % Simplified Chinese Support using external fonts (./fonts/zh_CN-Adobe/)
% \usepackage{NotoSansSC_external}
% \usepackage{NotoSerifCJKsc_external}
% \usepackage{zh_CN-Adobefonts_internal} % Simplified Chinese Support using system fonts
\usepackage{linespacing_fix} % disable extra space before next section
\usepackage{cite}

\begin{document}
\pagenumbering{gobble} % suppress displaying page number

\name{陈俊宏}

\basicInfo{
  \email{alal1995@hotmail.com} \textperiodcentered\ 
  \phone{(+86) 156-2232-1518} \textperiodcentered\ 
%   \linkedin[billryan8]{https://www.linkedin.com/in/billryan8}}
}

\section{\faUsers\ 项目经历}
\datedsubsection{\textbf{一对一系统开发 }   广州星火在线1对1} {2019.10 -- 至今}
\role{Golang后端开发}
\text{项目介绍:本项目是用于支撑星火在线一对一的在线直播业务,包含排课平台、订单、直播、运营增长平台,多个平台来支撑业务线的发展}
\begin{itemize}
   \item 实现了如下特性
   \begin{itemize}
     \item 业务相关开发
     \begin{enumerate}
       \item 负责运营系统功能开发(签单、退款、支付、增长模型部分、多省区数据权限功能、微信公众号)
       \item 上课系统业务系统开发(课件、排课系统)
       \item 优化业务系统部分复杂接口慢Sql优化(通过重构部分逻辑,使部分慢接口由大于1秒耗时,减少到 100ms,提升70\%)
     \end{enumerate}
   \end{itemize}
   \begin{itemize}
     \item 团队工具相关开发
     \begin{enumerate}
       \item 开发 Migration 命令行迁移工具,方便数据迁移和迁移记录审计。提升部署运维效率(部署时间从原来大于30分钟,到目前上版本约6分钟)。规范化上线代码流程
       \item 结合日志系统,实现链路追踪功能,提高团队debug效率(自己实现这套的而不使用 jaeger 进行实现是因为系统资源有限无法部署jaeger)
     \end{enumerate}
   \end{itemize}
   \begin{itemize}
     \item 阿里云 Kubernetes 相关使用经验
     \begin{enumerate}
       \item 结合 Jenkins 和 kubernetes 整合CI(持续集成流程), 通过编写 Groovy 脚本整合 CI PipeLine, 把原来每次发布的时间缩短了50\%
       \item 维护测试环境的应用(DeployMent) 发布和数据库(使用StatefulSet) ,结合阿里云云盘进行部署。稳定提供测试环境
     \end{enumerate}
   \end{itemize}
\end{itemize}

\begin{itemize}
  \item 取得成绩
  \begin{itemize}
    \item 团队贡献
    \begin{enumerate}
      \item 团队文档(1、推动功能设计文档进入研发SOP中,规范化开发流程; 2、编写 Git 团队使用文档(合并代码、发版规范、Commit格式,减少团队因为代码管理问题而产生线上问题)
      \item 编写脚本嵌入Git Pre-Commit Hook中,每次提交强制检查代码,减少代码低级错误
      \item 迁移 Go 版本(从1.12 到1.13),并且把项目整体依赖管理切换到 GoModule
    \end{enumerate}
    \item 业务支撑
    \begin{enumerate}
      \item 支持团队用户量从日总课次数从2019年 200个一对一课次上课到 目前日总课次高峰为 700个课次 
    \end{enumerate}
  \end{itemize}
  \item 技术栈:Golang, PostgreSQL, Redis, Gin, Jenkins, 阿里云相关云开发工具(监控、日志、Kubernetes、对象存储、Alimns消息队列)
\end{itemize}


\datedsubsection{\textbf{统一存储平台 } 深圳市杉岩数据技术有限公司} {2018.08 - 2019.07}
\role{Python后端开发}
\text{离职原因:因个人和家庭原因返回广州(家在广州)}
\newline
\text{项目介绍:本项目是用于提供更易用的运维管理平台用于 Ceph 存储, 主要提供块存储相关的功能}
\begin{onehalfspacing}
\begin{itemize}
  \item 实现了如下特性
  \begin{enumerate}
    \item  从 Ceph iSCSI 中获取性能统计功能
    \item  NAS(网络附加存储)的性能统计(包括nfs和cifs的协议的统计)
    \item 提供 Cephx 功能的对外操作api和ui(把原生Cephx的复杂的配置方法变成一键完成)
    \item 和节点回复的功能批量安装(不添加ssh公钥,对客户机器无侵入)
    \item 提升被依赖管理进程重试的速度,通过修改RCM(资源控制管理)进程中的状态机进行实现
  \end{enumerate}
  \item 取得的成绩
  \begin{enumerate}
    \item 重构了磁盘管理的功能,实现对用户层的存储介质(HDD,SSD,NVME SSD)的屏蔽,使得用户更加易用
    \item 部分职责:需要支持前端的同事(售前和运维)继续问题的处理,成功支持软件在多种公有云和私有云的部署以及问题的修复
  \end{enumerate}
\end{itemize}
\begin{itemize}
  \item 技术栈: RabbitMQ, Mariadb with galera cluster, Python, Django, graphite(TSDB的监控模块)
                Shell(主要支持安装、批量安装、卸载))
\end{itemize}
\end{onehalfspacing}

\datedsubsection{\textbf{Argus 监控系统 }  广州优亿信息科技有限公司} {2017.07 -- 2018.05}
\role{后端开发}
\text{离职原因:团队老大离职去腾讯,团队解散,不想转去其他团队去写Java}
\newline
\text{项目简介: 项目是为了提供全面的监控而设立的。提供基础架构级别的监控、应用级别的监控、分布式追踪级别的监控。并且使用了机器学习的技术来调整告警的阈值和降低误报率}
\begin{onehalfspacing}
\begin{itemize}
  \item 实现以下功能
  \begin{itemize}
    \item [1)]
     接手离职同事编写的 NodeJS 开发用户界面后端的逻辑
    \item [2)]
      实现大部分的监控告警的功能(包含了其他监控上有的基础功能之外,还实现了同比和环比的功能)
    \item [3)]
      添加了多个告警的推送方式(包括 Slack 和 微信)
    \item [4)]
      基于 sklearn 的算法,编写了适用于项目的AI训练流程(包括数据切片, 模型训练,模型认证,模型输出)
    \item [5)]
      整个大项目项目进行 docker 化,可以通过 docker-compose 即可进行简易的部署 
  \end{itemize}
  \item 获得成绩
  \begin{itemize}
    \item [1)]成功打入第一节 AIOPs 竞赛(由清华大学进行主办), 并且最终获得第10名的成绩。
    \item [2)]成功支撑电信的信号监控系统
  \end{itemize}
  \item 技术栈: Redis, MongoDB, Python, OPENTSDB, tcollector, NodeJS, sklearn
\end{itemize}
\end{onehalfspacing}

% Reference Test
%\datedsubsection{\textbf{Paper Title\cite{zaharia2012resilient}}}{May. 2015}
%An xxx optimized for xxx\cite{verma2015large}
%\begin{itemize}
%  \item main contribution
%\end{itemize}

\section{\faCogs\ IT 技能}
% increase linespacing [parsep=0.5ex]
\begin{itemize}[parsep=0.5ex]
  \item 编程语言: GO > Python > shell 
  \item 平台: Linux
  \item 其他相关技能:git, docker, jenkins, 阿里云使用经验(日志系统,kubernetes, alimns)
  \item 开发: Web, git, BackendDevelopment, DevOPS, 监控系统
\end{itemize}

\section{\faGraduationCap\  教育背景}
\datedsubsection{\textbf{广东工业大学}, 广州}{2013 -- 2017}
\textit{学士}\ 物联网工程

\section{\faInfo\ 其他}
% increase linespacing [parsep=0.5ex]
\textit{在业余时间完成 MIT6.824 分布式系统课程}
\begin{itemize}[parsep=0.5ex]
    \item 博客: http://ray1888.github.io 
    \item GitHub: https://github.com/ray1888
    \item 语言: 英语 - 熟练(六级 score:509), 普通话 - 流利, 粤语 - 流利
\end{itemize}

%% Reference
%\newpage
%\bibliographystyle{IEEETran}
%\bibliography{mycite}
\end{document}