% !TEX TS-program = xelatex
% !TEX encoding = UTF-8 Unicode
% !Mode:: "TeX:UTF-8"

\documentclass{resume}
\usepackage{zh_CN-Adobefonts_external} % Simplified Chinese Support using external fonts (./fonts/zh_CN-Adobe/)
% \usepackage{NotoSansSC_external}
% \usepackage{NotoSerifCJKsc_external}
% \usepackage{zh_CN-Adobefonts_internal} % Simplified Chinese Support using system fonts
\usepackage{linespacing_fix} % disable extra space before next section
\usepackage{cite}

\begin{document}
\pagenumbering{gobble} % suppress displaying page number

\name{陈俊宏}

\basicInfo{
  \email{alal1995@hotmail.com} \textperiodcentered\ 
%   \phone{(+86) 131-221-87xxx} \textperiodcentered\ 
%   \linkedin[billryan8]{https://www.linkedin.com/in/billryan8}}
}

\section{\faUsers\ 项目经历}

\datedsubsection{\textbf{统一存储平台}}{2018.08 -- 2019.07}
\company{深圳市杉岩数据技术有限公司} \textperiodcentered\ 
\role{后端开发}{组长: 杨德柳}
简介: 使用Python开发基于Ceph的管理平台,对外提供二次发行版的Ceph,提供了给用户简单易用的UI而不用基于命令行进行管理。平台支持了Ceph的部署,存储网关部署、以及相关的监控(服务、进程、性能级别的监控)
    \subsubsection{\textbf {实现了如下特性}}
    \begin{itemize}
        \item [1)]
          添加监控收集指标(1.Nas[网络附加存储]的性能统计[包括nfs和cifs的协议的统计] 2. collect iSCSI Performance with ceph client )
        \item [2)]
          实现一键打开CephX功能的API(包括了配置修改、密钥创建并且相关业务进程的重启)
        \item [3)]
        批量安装(不添加ssh公钥,对客户机器无侵入, 并且提供并发控制)和节点回复的功能(用于硬件故障或者系统重装后的配置的重新生成并且加入集群)
    \end{itemize}
    \subsubsection{\textbf {取得的成绩}}
    \begin{itemize}
        \item [1)]
            重构了磁盘管理的功能,实现对用户层的存储介质的屏蔽,使得用户更加易用
        \item [2)] 
            部分职责:需要支持前端的同事(售前和运维)继续问题的处理,成功支持软件在多种公有云和私有云的部署以及问题的修复
    \end{itemize}
    \begin{itemize}
    \item 技术栈: RabbitMQ, Mariadb with galera cluster, Python, Django, graphite(TSDB的监控模块)
    Shell(主要支持安装、批量安装、卸载))
    \end{itemize}

\datedsubsection{\textbf{Argus监控系统}}{2017.07 -- 2018.04}
\company{广州优亿信息科技有限公司} \textperiodcentered\ 
\role{后端开发}{组长: 汤英康}
项目简介: 项目是为了提供全面的监控而设立的。提供基础架构级别的监控、应用级别的监控、分布式追踪级别的监控。并且使用了机器学习的技术来调整告警的阈值和降低误报率
    \subsubsection{\textbf {实现了如下特性}}
    \begin{itemize}
        \item [1)]
            使用node.js接手开发用户界面后端的逻辑和Dashboard的接口
        \item [2)]
            实现大部分的监控告警的功能(包含了其他监控上有的基础功能之外,还实现了同比和环比的功能, 并且支持告警推送到Slack和微信)
        \item [3)]   
            基于sklearn的算法,编写了适用于项目的AI训练流程(包括数据切片, 模型训练,模型认证,模型输出) 
        \item [4)] 
            整个大项目项目进行docker化,可以通过docker-compose 即可进行简易的部署
    \end{itemize}
    \subsubsection{\textbf {取得的成绩}}
    \begin{itemize}
        \item [1)]
            与同期稳定版的Bosun项目相比,我们的监控系统可以支持热更新监控规则,而Bosun需要重启并且使用脚本重新生成告警配置。
        \item [2)] 
            成功打入第一节AIOPs竞赛(由清华大学进行主办), 并且最终获得第10名的成绩。
    \end{itemize}
    \begin{itemize}
    \item 技术栈: Redis, Mongodb, Python, OPENTSDB, tcollector, node.js, sklearn
    \end{itemize}

\section{\faCogs\ IT 技能}
% increase linespacing [parsep=0.5ex]
\begin{itemize}[parsep=0.5ex]
  \item 编程语言: Python,GO > shell > JavaScript
  \item 平台: Linux
  \item 其他相关技能:git, docker, jenkins
  \item 开发: Web, git, BackendDevelopment, DevOPS
\end{itemize}

\section{\faGraduationCap\  教育背景}
\datedsubsection{\textbf{广东工业大学}, 广州}{2013 -- 2017}
\textit{学士}\ 物联网工程

\section{\faHeartO\ 业余学习}
\textit{完成MIT 6.824 分布式系统课程}

\section{\faInfo\ 其他}
% increase linespacing [parsep=0.5ex]
\begin{itemize}[parsep=0.5ex]
    \item 博客: http://ray1888.github.io 
    \item GitHub: https://github.com/ray1888
    \item 语言: 英语 - 熟练(六级 score:509), 普通话 - 流利, 粤语 - 流利
\end{itemize}

%% Reference
%\newpage
%\bibliographystyle{IEEETran}
%\bibliography{mycite}
\end{document}