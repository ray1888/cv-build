% !TEX TS-program = xelatex
% !TEX encoding = UTF-8 Unicode
% !Mode:: "TeX:UTF-8"

\documentclass{resume}
\usepackage{zh_CN-Adobefonts_external} % Simplified Chinese Support using external fonts (./fonts/zh_CN-Adobe/)
% \usepackage{NotoSansSC_external}
% \usepackage{NotoSerifCJKsc_external}
% \usepackage{zh_CN-Adobefonts_internal} % Simplified Chinese Support using system fonts
\usepackage{linespacing_fix} % disable extra space before next section
\usepackage{cite}

\begin{document}
\pagenumbering{gobble} % suppress displaying page number

\name{陈俊宏}

\basicInfo{
  \email{alal1995@hotmail.com} \textperiodcentered\ 
  \phone{(+86) 156-2232-1518} \textperiodcentered\ 
%   \linkedin[billryan8]{https://www.linkedin.com/in/billryan8}}
}

\section{\faUsers\ 项目经历}
\datedsubsection{\textbf{一对一系统开发} 广州星火在线1对1 }{2019年10月 -- 至今}
\role{Golang 后端开发}
简介:
\begin{itemize}
  \item 职能开发
    \item [1)] 
    负责运营系统功能开发(签单、退款、支付、增长模型部分、多省区数据权限功能、微信公众号)
    \item [2)]
    上课系统业务系统开发(课件、排课系统)
  \item 团队工具及性能优化
    \item [1)] 
    迁移 Go 版本(从1.12 到1.13),并且把项目整体依赖管理切换到 GoModule
    \item [2)] 
    开发 Migration 命令行迁移工具,方便数据迁移和迁移记录审计。提升部署运维效率(部署时间从原来大于30分钟,到目前上版本约6分钟)
    \item [3)] 
    使用 Jenkins + 阿里云托管 Kubernets 构建CI体系,支持项目快速迭代。同时规范化上线代码流程
    \item [4)] 
    业务系统部分复杂接口慢Sql优化(通过重构部分逻辑,使部分慢接口由大于1秒耗时,减到小于 300ms,提升70\%)
    \item [5)] 
    结合日志系统,实现链路追踪功能,提高团队debug效率(自己实现这套的而不使用 jaeger 进行实现是因为系统没有多余资源部署jaeger)
  \item 团队贡献
   \item [1)] 
      推动功能设计文档进入研发SOP中,规范化开发流程
   \item [2)]
      编写 Git 团队使用文档(合并代码、发版规范、Commit格式,减少团队因为代码管理问题而产生线上问题)
   \item [3)]
      编写脚本嵌入Git Pre-Commit Hook中,每次提交强制检查代码,减少代码低级错误
\end{itemize}

\datedsubsection{\textbf{统一存储平台} 深圳市杉岩数据技术有限公司 深圳}{2018年8月 -- 2019年7月}
\role{Python 后端开发}
简介:
\begin{itemize}
  \item 实现了如下特性
    \item [1)] 
    从Ceph iSCSI 中获取性能统计并且展示到页面上
    \item [2)]
    Nas(网络附加存储)的性能统计(包括nfs和cifs的协议的统计)
    \item [3)]
    提供Cephx功能的对外操作api和ui(把原生Cephx的复杂的配置方法变成一键完成)
    \item [4)]
    批量安装(不添加ssh公钥,对客户机器无侵入)和节点回复的功能
    \item [5)]
    提升被依赖管理进程重试的速度,通过修改RCM(资源控制管理)进程中的状态机进行实现
  \item 重构了磁盘管理的功能,实现对用户层的存储介质的屏蔽,使得用户更加易用
  \item 部分职责:需要支持前端的同事(售前和运维)继续问题的处理,成功支持软件在多种公有云和私有云的部署以及问题的修复
  \item 技术栈: RabbitMQ, Mariadb with galera cluster, Python, Django, graphite(TSDB的监控模块)
                Shell(主要支持安装、批量安装、卸载))
\end{itemize}
\end{onehalfspacing}

% Reference Test
%\datedsubsection{\textbf{Paper Title\cite{zaharia2012resilient}}}{May. 2015}
%An xxx optimized for xxx\cite{verma2015large}
%\begin{itemize}
%  \item main contribution
%\end{itemize}

\section{\faCogs\ IT 技能}
% increase linespacing [parsep=0.5ex]
\begin{itemize}[parsep=0.5ex]
  \item 编程语言: GO > Python > shell 
  \item 平台: Linux
  \item 其他相关技能:git, docker, jenkins, 阿里云使用经验(日志系统,kubernetes, alimns)
  \item 开发: Web, git, BackendDevelopment, DevOPS, 监控系统
\end{itemize}

\section{\faGraduationCap\  教育背景}
\datedsubsection{\textbf{广东工业大学}, 广州}{2013 -- 2017}
\textit{学士}\ 物联网工程

\section{\faHeartO\ 获奖情况}
\textit{打入第一届清华大学举办 AIOps 竞赛的全国第十名}
\textit{完成 MIT6.824 分布式系统课程}

\section{\faInfo\ 其他}
% increase linespacing [parsep=0.5ex]
\begin{itemize}[parsep=0.5ex]
    \item 博客: http://ray1888.github.io 
    \item GitHub: https://github.com/ray1888
    \item 语言: 英语 - 熟练(六级 score:509), 普通话 - 流利, 粤语 - 流利
\end{itemize}

%% Reference
%\newpage
%\bibliographystyle{IEEETran}
%\bibliography{mycite}
\end{document}