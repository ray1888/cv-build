% !TEX program = xelatex

\documentclass{resume}
%\usepackage{zh_CN-Adobefonts_external} % Simplified Chinese Support using external fonts (./fonts/zh_CN-Adobe/)
%\usepackage{zh_CN-Adobefonts_internal} % Simplified Chinese Support using system fonts

\begin{document}
\pagenumbering{gobble} % suppress displaying page number

\name{JunHong Chen}

\basicInfo{
  \email{alal1995@hotmail.com} \textperiodcentered\ 
%   \phone{(+86) 131-221-87xxx} \textperiodcentered\ 
%   \linkedin[billryan8]{https://www.linkedin.com/in/billryan8}
}

\section{\faUsers\ Experience}
\datedsubsection{\textbf{Unifly Storage Platform}}{Aug 2018 -- Jul 2019}
\company{Sandstone Co.,LTD, Shenzhen, China} \textperiodcentered\ 
\role{PythonBackEndDev, DevOP}{Manager: Deliu Yang}
Brief introduction: Developing with management platform With Python for commercial re-packaged Ceph Storage System as a service (Saas) for Deploying, Monitoring  re-developed distribution of Ceph
    \subsubsection{\textbf {Implemented feature}}
    \begin{itemize}
        \item [1)]
          Monitor Collect Enhancement(1.add NAS[network attach storage] Performance Monitor [including nfs and cifs protocol support] 2. collect iSCSI Performance with ceph client )
        \item [2)]
          Repackage Cephx Management in web base api (including do cephx config enable and process restart in one button)
        \item [3)]
          Batch install (without add ssh key with concurrency control) and Node recovery(for recover the node after node failure, hardware failed or system failed)
    \end{itemize}
    \subsubsection{\textbf {Archivement and Enhancement}}
    \begin{itemize}
        \item [1)]
          The resource failure and restart speed with optimize the RCM(Resource Control Manager) process FSM
        \item [2)] 
          Remote Handling deploy and usage problem in various environment(included public cloud and private cloud) for on-site ops engineer
    \end{itemize}
    \begin{itemize}
    \item Tech Stack: RabbitMQ, Mariadb with galera cluster, Python, Django, graphite(TSDB Monitor Compoenet) Shell(Install and Batch install script and remove script)
    \end{itemize}


\datedsubsection{\textbf{Argus Monitor Project}}{Jul 2017 -- Apr.2018}
\company{Useease Co.LTD, Guangzhou, China}
\role{Maintainer, BackEndDev}{OpenSource Projects, collaborated with Team Argus}
Brief introduction: A Monitor system build for all rounded usage for infra monitor, application monitor, distributed tracing monitor And Using machine learn to adjust the threhold and optimize the wrong alert rate.
    \subsubsection{\textbf {Implemented feature}}
    \begin{itemize}
        \item [1)]  
          Maintain and develop the Web base backend with node.js and dashboard 
        \item [2)]
          Implement the Majority part of Alert System (including the base rule, also add Compared with week feature and support push alert by Slack and Wechat)
        \item [3)]
          Write the basic ai train framework with sklearn (slice data into training and verifcation, trian model, evaluate model precision)
        \item [4)]
          Dockerize the project with multi-dockerfile and docker-compose 
    \end{itemize} 
    \subsubsection{\textbf {Archivement and Enhancement}}
    \begin{itemize}
        \item [1)]
        Compare to early state bosun, we can change monitoring rules without restart the process 
        \item [2)] 
        Successfully Entry the First AIOps Challenage (Held by tsinghua university) and get the 10th of the competition
    \end{itemize}
    \begin{itemize}
    \item Tech Stack: Redis, Mongodb, Python, OpenTSDB, tcollector, node.js, sklearn
    \end{itemize}


\section{\faCogs\ Skills}
\begin{itemize}[parsep=0.5ex]
  \item Programming Languages: Python, GO > shell > JavaScript
  \item Platform: Linux
  \item Other: git, docker, jenkins
  \item Development: Web, git, BackendDevelopment, DevOPS
\end{itemize}

\section{\faGraduationCap\ Education}
\datedsubsection{\textbf{Guangdong University of Technology (GDUT)}, guangzhou, China}{2013 -- 2017}
\textit{B.S.} in Internet of Thing

\section{\faHeartO\ Programming Apart from work }
\begin{itemize} 
   \item Finish MIT 6.824 Distributed System Course 
\end {itemize}

\section{\faInfo\ Miscellaneous}
\begin{itemize}[parsep=0.5ex]
  \item Blog: http://ray1888.github.io 
  \item GitHub: https://github.com/ray1888
  \item Languages: English - Fluent(CET-6 score:509), Mandarin - Native speaker, Cantonese - Native speaker
\end{itemize}

%% Reference
%\newpage
%\bibliographystyle{IEEETran}
%\bibliography{mycite}
\end{document}